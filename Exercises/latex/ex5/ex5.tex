%% You can use this file to create your answer for Exercise 5.  
%% Fill in the places labeled by comments.
%% Generate a PDF document by with the command `pdflatex ex5'.

\documentclass[11pt]{article}


\usepackage{times}
\usepackage{listings}
\usepackage{enumerate}
\usepackage{courier}
\usepackage{hyperref}
\usepackage{xcolor}
\usepackage{graphicx}


%% Values that are specific to a particular term
\newcommand{\thisterm}{Spring 2020}

\newcommand{\dateassigned}{Mon.,~March 30}

%% Printed form of home page that students should use
\newcommand{\visiblecoursehome}{http://www.cs.cmu.edu/\textasciitilde{}418}

%% Link to home page that will stay valid
\newcommand{\actualcoursehome}{http://www.cs.cmu.edu/afs/cs.cmu.edu/academic/class/15418-s20/www}

\newcommand{\datedue}{Monday.,~April~6}





%% Page layout
\oddsidemargin 0pt
\evensidemargin 0pt
\textheight 600pt
\textwidth 469pt
\setlength{\parindent}{0em}
\setlength{\parskip}{1ex}

%% Colored hyperlink 
\newcommand{\cref}[2]{\href{#1}{\color{blue}#2}}

%% Customization to listing
\lstset{basicstyle=\ttfamily\small,language=C,morekeywords={boolean,xbegin,xabort,xend}}

%% Enumerate environment with alphabetic labels
\newenvironment{choice}{\begin{enumerate}[A.]}{\end{enumerate}}
\newenvironment{subchoice}{\begin{enumerate}[(1)]}{\end{enumerate}}
%% Environment for supplying answers to problem
\newenvironment{answer}{\begin{minipage}[c][2.0in]{\textwidth}}{\end{minipage}}


\begin{document}
\begin{flushright}                         
{\large\bf Full Name: \makebox[2in][l]{    
%% Put your name on the next line          
                                           
                                           
}} \\[1ex]                                 
                                           
{\large\bf Andrew Id: \makebox[2in][l]{\tt 
%% Put your Andrew ID on the next line     
                                           
}} \\[1ex]                                 
\end{flushright}                           
\vspace*{0.3in}                            
\begin{center}
\LARGE
15-418/618 \thisterm{} \\
Exercise 5
\end{center}

\begin{center}
\Large        
\begin{tabular}{ll}
\hline             
Assigned: & \dateassigned{}  \\
Due: &  \datedue{}, 11:00~pm  \\
\hline       
\end{tabular}
\end{center} 

\section*{Overview}

This exercise is designed to help you better understand the lecture
material and be prepared for the style of questions you will get on
the exams.  The questions are designed to have simple answers.  Any
explanation you provide can be brief---at most 3 sentences.  You
should work on this on your own, since that's how things will be when
you take an exam.

You will submit an electronic version of this assignment to Gradescope 
as a PDF file.  For those of you familiar with the \LaTeX{} text 
formatter, you can download the template, configuation, and figure files at: 
\begin{center} 
  \cref{\actualcoursehome/exercises/config-ex5.tex}{\visiblecoursehome/exercises/config-ex5.tex}\\ 
  \cref{\actualcoursehome/exercises/ex5.tex}{\visiblecoursehome/exercises/ex5.tex} \\
\end{center} 
Instructions for how to use this template are included as comments in 
the file.  Otherwise, you can use this PDF document as your starting 
point.  You can either: 1) electronically modify the PDF, or 2) print 
it out, write your answers by hand, and scan it.  In any case, we 
expect your solution to follow the formatting of this document. 

\newpage

\section*{Problem 1: Locking and Transactional Memory}


Consider the following very simple banking system, where all account values are whole dollars.
\begin{lstlisting}
// Account balances in whole dollars
int balance[NACCT];
\end{lstlisting}

The only supported banking operations are to transfer money between accounts
and to check that the sum of all balances equals or exceeds some threshold value.

\begin{lstlisting}
// Transfer money between accounts.
// Do not allow from_acct to become negative
boolean transfer(int to_acct, int from_acct, int amount);

// Check that sum of all accounts equals or exceeds some threshold
boolean check_sum(int threshold);
\end{lstlisting}

If a transfer operation would cause an account to become overdrawn,
function \texttt{transfer} should return false without changing any balances.  (You can
assume that {\tt amount} is not negative.)

Function \verb@check_sum@ must be implemented in such a way that it
gets a value for the sum that would be a valid snapshot of the
state of the balances for some sequentially consistent ordering of the
transactions.

\newpage
\subsection*{1A: A Banking System with Locks}

Suppose
you are provided a mutual exclusion lock for each account:
\begin{lstlisting}
int lock[NACCT];
\end{lstlisting}

The library functions for acquiring and releasing a lock have the following prototypes:
\begin{lstlisting}
void acquire(int *lockp);
void release(int *lockp);  
\end{lstlisting}

Fill in code for these two operations using locking.  You may not add
any additional locks.  Make sure your code cannot deadlock.  Try to
minimize the amount of time that locks are held.

\begin{enumerate}
\item Transfer
%% Fill in the lines below with your code for 1A-1
\begin{lstlisting}
boolean transfer(int to_acct, int from_acct, int amount) {
    boolean ok = true;
    acquire(lock + from_acct);
    if (balance[from_acct] < amount)
    {
      ok = false;
    }
    else
    {
      balance[from_acct] -= amount;
    }
    release(lock + from_acct);

    if (ok)
    {
      acquire(lock + to_acct);
      balance[to_acct] += amount;
      release(lock + to_acct);
    }

    return ok;
}
\end{lstlisting}

\item Check Sum
%% Fill in the lines below with your code for 1A-2
\begin{lstlisting}
boolean check_sum(int threshold) {
    int sum = 0;
    for (Int i = 0; i < NACCT; i++)
    {
      acquire(lock + i);
      sum += balance[i];
      release(lock + i);
    }

    return sum >= threshold;
}
\end{lstlisting}
\end{enumerate}

\newpage
\subsection*{1B: A Banking System with Transactional Memory}

Suppose we have a transactional memory system supporting the following operations:
\begin{lstlisting}
// Begin transaction
void xbegin();

// Abort transaction
void xabort();

// (Attempt to) commit transaction.
// Returns true if successful, false if failed
boolean xend();
\end{lstlisting}

Rewrite the two functions using transactional memory primitives rather than locks.

\begin{enumerate}
\item Transfer
%% Fill in the lines below with your code for 1B-1
\begin{lstlisting}
boolean transfer(int to_acct, int from_acct, int amount) {
    boolean ok;
    while (true)
    {
      xbegin();
      ok = balance[from_acct] >= amount;
      if (ok)
      {
        balance[from_acct] -= amount;
        balance[to_acct] += amount;
        if (xend())
        {
          break;
        }
      }
      else
      {
        xabort();
      }
    }

   return ok;
}
\end{lstlisting}

\item Check Sum
%% Fill in the lines below with your code for 1B-2
\begin{lstlisting}
boolean check_sum(int threshold) {
    int sum = 0;
    while (true)
    {
      sum = 0;
      xbegin();
      for (int i = 0; i < NACCT; i++)
      {
        sum += balance[i];
      }
      if (xend())
      {
        break;
      }
    }

    return sum >= threshold;
}
\end{lstlisting}
\item If the system performs millions of transactions per second over
  thousands of bank accounts, what problem to you forsee for the  implementation based on transactional memory?
  Assume the transactional memory is implemented using lazy versioning and optimistic conflict detection.

\begin{answer}
%% Write your answer for 1B-3 here

\textcolor{red}{Chances are that \lstinline{check_sum} would keep failing, because ongoing 
transfers would keep causing writes to the array \lstinline{balance}.}

\textcolor{cyan}{Lazy versioning means that it won't write to memory directly, namely it won't 
create too much memory write operations. Optimistic conflict detection means that it only detects
conflict on \lstinline{xend()}. \lstinline{check_sum} need a lot of time to operate on 
\lstinline{balance}, while \lstinline{transfer} only need to operate two elements. So when 
\lstinline{check_sum} tries to commit, it often will find that some of elements is been
modified, and hence failed. }

\end{answer}
\end{enumerate}

\newpage

\subsection*{1C: Improving transactional memory performance}

Suppose that you can be guaranteed that there will be only one
outstanding call to \texttt{check\_sum} at a time, and that it will be
relatively infrequent (around once per hour.)  Devise a simple scheme, without any locks,
where the system can temporarily force all transfers to wait
while a sum is being computed, thereby minimizing the
chance of the sum computation failing.  You may make use of the following
additional global variable:

\begin{lstlisting}
boolean summing = false;
\end{lstlisting}

\begin{enumerate}
\item Transfer
%% Fill in the lines below with your code for 1C-2
\begin{lstlisting}
boolean transfer(int to_acct, int from_acct, int amount) {
    boolean ok;
    while (true)
    {
      xbegin();
      ok = balance[from_acct] >= amount && !summing;
      if (ok)
      {
        balance[from_acct] -= amount;
        balance[to_acct] += amount;
        if (xend())
        {
          break;
        }
      }
      else
      {
        xabort();
      }
    }

   return ok;
}
\end{lstlisting}

\item Check Sum
%% Fill in the lines below with your code for 1C-3
\begin{lstlisting}
boolean check_sum(int threshold) {
    int sum = 0;
    summing = true;
    while (true)
    {
      sum = 0;
      xbegin();
      for (int i = 0; i < NACCT; i++)
      {
        sum += balance[i];
      }
      if (xend())
      {
        break;
      }
    }
    summing = false;

    return sum >= threshold;
}
\end{lstlisting}
\end{enumerate}

\newpage

\section*{Problem 2: Processor Scaling}

This problem explores the two scaling models presented on slides 5--9 in the lecture on Heterogeneous Parallelism:
\begin{center}
\cref{\actualcoursehome/lectures/19_heterogeneity.pdf}{\visiblecoursehome/lectures/21\_heterogeneity.pdf}.
\end{center}
These are based on the paper ``Amdahl's Law in the Multicore Era,'' by
Mark D. Hill and Michael R. Marty of the University of Wisconsin,
published in {\em IEEE Computer}, July, 2008.

In both models, we express computing performance and resources (e.g.,
chip area) relative to a baseline processor.  In scaling to a
processor with $r$ resource units (scaled such that the baseline
processor has $r=1$), we obtain a processor with single-threaded
performance ${\it perf}(r)$ (scaled such ${\it perf}(1) = 1$.)

The slides show graphs for the case of ${\it perf}(r) = \sqrt{r}$.
This function is plausible---it captures the idea that increasing
processing resources will improve performance, but not in a linear
way.  Increasing to large values of $r$ yields diminishing returns in
terms of single-threaded processor performance.  We will generalize
this model to one with ${\it perf}(r) = r^{\alpha}$, where $\alpha$ is
value with $0.0 \leq \alpha \leq 1.0$.  (For example, the slides are
based on $\alpha = 0.5$.)

For scaling the system design to use $n$ resource units, two
options are considered:
\begin{description}
\item[Homogeneous:] We create a new processor design that uses $r$
  resource units, and populate the system with $p = n/r$ identical
  processors.  In this case, $p$ must be an integer, but $r$ need not
  be.  For example, to get $p=5$ when $n=16$, we would have $r = 16/5 = 3.2$.

\item[Heterogeneous:] We create a design that uses $r$ resource units (where $r$ is an integer),
  and then create a system with one of these more powerful processors,
  plus $n-r$ baseline processors, giving $p = n - r + 1$ total processors.
\end{description}

As with the conventional presentation of Amdahl's Law, we assume that
the problem to be solved has some fraction $f$ that can use arbitrary
levels of parallelism, while the remaining fraction $1-f$ must be
executed sequentially.

With performance normalized to a single baseline processor, Amdahl's Law
states that the speedup over the baseline is given by the equation
\begin{eqnarray}
S & = & \frac{1}{T_{\rm seq} + T_{\rm par}} \label{eqn:amdahl:basic}
\end{eqnarray}
where $T_{\rm seq}$ is the time required to execute the sequential
portion of the code and $T_{\rm par}$ is the time required to execute
the parallel portion of the code, with both of these using all
available resources.

\newpage

\subsection*{2A: Homogeneous Model Speedup}
Slide 6 gives the following equation for the speedup in the homogeneous model:
\begin{eqnarray}
S_{\rm ho} & = & \frac{1}{\frac{1-f}{{\it perf}(r)} + \frac{f}{{\it
      perf}(r) \cdot  \frac{n}{r}}} \label{eqn:amdahl:homogeneous}
\end{eqnarray}

Explain how Equation \ref{eqn:amdahl:homogeneous} follows the form of
Equation \ref{eqn:amdahl:basic}.
\begin{enumerate}
\item
What resources are used and what is the
duration of $T_{\rm seq}$?

\begin{answer}
%% Write your answer to Problem 2A-1 here

The one big processor with performance $perf(r)$ is used, and $T_{\rm seq}=\frac{1-f}{perf(r)}$

\end{answer}
\item
 What resources are used and what is the
duration of $T_{\rm par}$?

\begin{answer}
%% Write your answer to Problem 2A-2 here

Those $\frac{n}{r}$ baseline processors with performance $perf(r)$ are used, 
and $T_{\rm par}=\frac{f}{perf(r)\cdot\frac{n}{r}}$

\end{answer}
\end{enumerate}

\newpage
\subsection*{2B: Heterogeneous Model Speedup}
Slide 8 gives the following equation for the speedup in the heterogenous model:
\begin{eqnarray}
S_{\rm he} & = & \frac{1}{\frac{1-f}{{\it perf}(r)} + 
\frac{f}{{\it perf}(r) +  n -r}}  \label{eqn:amdahl:heterogeneous}
\end{eqnarray}

Explain how Equation \ref{eqn:amdahl:heterogeneous} follows the form
of Equation
\ref{eqn:amdahl:basic}.
\begin{enumerate}
\item
What resources are used and what is the duration of $T_{\rm seq}$?

\begin{answer}
%% Write your answer to Problem 2B-1 here

The one bit processor with performance $perf(r)$ is used, and $T_{\rm seq}=\frac{1-f}{perf(r)}$

\end{answer}


\item
What resources are used and what is the duration of $T_{\rm par}$?

\begin{answer}
%% Write your answer to Problem 2B-2 here

The one processor with performance $perf(r)$ and $n-r$ processors with performance $perf(1)$ is 
used, and $T_{\rm par}=\frac{f}{perf(r)+n-r}$. 

\end{answer}
\end{enumerate}

\newpage

\subsection*{2C: Optimizing Parameter $r$ for the Homogeneous Model}

For either of these models, we can find a value $r^{*}$ that achieves
maximum speedup as a function of $f$, $n$, and $\alpha$.  Keeping $n$
fixed at 256 but varying parameters $\alpha$ and $f$, fill in the
table below giving the values of $r^{*}$ and $p = n/r^{*}$ for the
homogeneous model, and the resulting speedups.

{\bf Hint:} The number of processors $p = n/r$ must be an integer
ranging from 1 to $n$.  You can
simply compute
\ref{eqn:amdahl:homogeneous} for all possible values and select the
one with maximum speedup.

You may find the following spreadsheet a useful starting point:
\begin{center}
\cref{\actualcoursehome/exercises/ex5-model.xlsx}{\visiblecoursehome/exercises/ex5-model.xlsx}
\end{center}
This spreadsheet generates the graphs shown on slide 9 for the case of
$n=256$.  You can adapt this spreadsheet to determine the values of
$r^{*}$.  (The spreadsheet only considers only cases where $p$ is a power of two.
You must consider other integer values.)

\begin{center}
\renewcommand{\arraystretch}{2.0}
\begin{tabular}{|cc|c|c|c|}
\hline
\makebox[0.5in]{$\alpha$} & 
\makebox[0.5in]{$f$} & 
\multicolumn{1}{|c}{\makebox[1.0in]{$r^{*}$}} & 
\multicolumn{1}{c}{\makebox[1.0in]{$p = n/r^{*}$}} & 
\multicolumn{1}{c|}{\makebox[1.0in]{$S_{\rm ho}$}} \\
\hline
0.5 & 0.700 & 
%% Give the value of r* for alpha = 0.5, f = 0.700
128.0
&
%% Give the value of p = n/r* for alpha = 0.5, f = 0.700
2
&
%% Give the value of S_ho for alpha = 0.5, f = 0.700
17.41
\\
\cline{3-5}
0.5 & 0.970 & 
%% Give the value of r* for alpha = 0.5, f = 0.970
8.0
&
%% Give the value of p = n/r* for alpha = 0.5, f = 0.970
32
&
%% Give the value of S_ho for alpha = 0.5, f = 0.970
46.90
 \\
\cline{3-5}
0.5 & 0.995 & 
%% Give the value of r* for alpha = 0.5, f = 0.995
1.29
&
%% Give the value of p = n/r* for alpha = 0.5, f = 0.995
199

&
%% Give the value of S_ho for alpha = 0.5, f = 0.995
113.42
 \\
\hline
0.7 & 0.700 & 
%% Give the value of r* for alpha = 0.7, f = 0.700
256.0
&
%% Give the value of p = n/r* for alpha = 0.7, f = 0.700
1
&
%% Give the value of S_ho for alpha = 0.7, f = 0.700
48.50
 \\
\cline{3-5}
0.7 & 0.970 & 
%% Give the value of r* for alpha = 0.7, f = 0.970
18.29
&
%% Give the value of p = n/r* for alpha = 0.7, f = 0.970
14
&
%% Give the value of S_ho for alpha = 0.7, f = 0.970
77.02
 \\
\cline{3-5}
0.7 & 0.995 & 

%% Give the value of r* for alpha = 0.7, f = 0.995
3.01
&
%% Give the value of p = n/r* for alpha = 0.7, f = 0.995
85
&
%% Give the value of S_ho for alpha = 0.7, f = 0.995
129.51
 \\
\hline
\end{tabular}
\end{center}

\newpage

\subsection*{2D: Optimizing Parameter $r$ for the Heterogeneous Model}

Fill in the table below giving the values of $r^{*}$ and $p =
n-r^{*}+1$ for the Heterogeneous model, and the resulting speedups.

\begin{center}
\renewcommand{\arraystretch}{2.0}
\begin{tabular}{|cc|c|c|c|}
\hline
\makebox[0.5in]{$\alpha$} & 
\makebox[0.5in]{$f$} & 
\multicolumn{1}{|c}{\makebox[1.0in]{$r^{*}$}} & 
\multicolumn{1}{c}{\makebox[1.0in]{$p = n - r^{*} + 1$}} & 
\multicolumn{1}{c|}{\makebox[1.0in]{$S_{\rm he}$}} \\
\hline
0.5 & 0.700 & 
%% Give the value of r* for alpha = 0.5, f = 0.700
170
&
%% Give the value of p = n - r* + 1 for alpha = 0.5, f = 0.700
87
&
%% Give the value of S_ho for alpha = 0.5, f = 0.700
33.25
\\
\cline{3-5}
0.5 & 0.970 &
%% Give the value of r* for alpha = 0.5, f = 0.970
72
&
%% Give the value of p = n - r* + 1 for alpha = 0.5, f = 0.970
185
&
%% Give the value of S_ho for alpha = 0.5, f = 0.970
116.62
 \\
\cline{3-5}
0.5 & 0.995 & 
%% Give the value of r* for alpha = 0.5, f = 0.995
28
&
%% Give the value of p = n - r* + 1 for alpha = 0.5, f = 0.995
229
&
%% Give the value of S_ho for alpha = 0.5, f = 0.995
191.94
 \\
\hline
0.7 & 0.700 & 
%% Give the value of r* for alpha = 0.7, f = 0.700
163
&
%% Give the value of p = n - r* + 1 for alpha = 0.7, f = 0.700
94
&
%% Give the value of S_ho for alpha = 0.7, f = 0.700
71.75
 \\
\cline{3-5}
0.7 & 0.970 & 
%% Give the value of r* for alpha = 0.7, f = 0.970
65
&
%% Give the value of p = n - r* + 1 for alpha = 0.7, f = 0.970
192
&
%% Give the value of S_ho for alpha = 0.7, f = 0.970
160.18
 \\
\cline{3-5}
0.7 & 0.995 & 

%% Give the value of r* for alpha = 0.7, f = 0.995
27
&
%% Give the value of p = n - r* + 1 for alpha = 0.7, f = 0.995
230
&
%% Give the value of S_ho for alpha = 0.7, f = 0.995
214.59
 \\
\hline
\end{tabular}
\end{center}

\newpage

\section*{Problem 3: Cache Performance of Multi-Threaded Program (MODIFIED)}

This problem relates to the image processing task described in Slides
40--46 of Lecture 22:
\begin{center}
\cref{\actualcoursehome/lectures/22_dsl.pdf}{\visiblecoursehome/lectures/22\_dsl.pdf}.
\end{center}

You have been hired by a movie production company to write programs
used for rendering images.  Each frame is $h \times w$ pixels, with a
pixel represented with data type \texttt{float}.  Your first task is
to write filtering programs that apply $k \times k$ filters to an
image.  Code to do this is shown for on Slide 42 for the case of
$k=3$.  The filters you must implement can be divided into two phases:
one horizontal filtering stage and one vertical stage, as shown in
Slide 43.

Consider the code of Slide 43.  In this version, you
allocate a temporary buffer large enough to hold all $w \times
(h+k-1)$ pixels of the result generated by the first phase and used by
the second.  The total work performed is proportional to $2k
\cdot w \cdot h$.  We call this the {\em baseline} code.

In the ideal case, the cache would be large enough to generate the
temporary buffer and keep it resident in cache while it is being
consumed in the second phase.
Unfortunately, that is not feasible.  You have to work within the following constraints:
\begin{itemize}
\item The images are 4K $\times$ 4K, i.e., $w = h = 4096$.
\item The machine has 16 cores sharing a single 2~MB ($2^{21}$-byte)
cache.
\end{itemize}

You therefore consider the code of Slide 46, where you choose a {\em
  chunk} size $c$, and size the temporary buffer to hold $c + k-1$
rows.  The code then processes the image in multiple {\em passes}.
Each pass has two phases, with the first filling the temporary buffer
based on $c+k-1$ rows of the input, and the second generating $c$ rows
of the output.  By selecting a small enough chunk size, the temporary
buffer will be stored in the cache during phase 1 and remain in the
cache as it is being used in phase 2.

There are many images to process, and so you plan to use the multiple
cores to run multiple threads, each operating on separate images.
However, you want to select a chunk size $c$ such that all of the
threads will keep their temporary buffers resident in cache.

\newpage

\begin{choice}
\item
Consider the single-threaded case.  Your colleague states that, as
long as the chunk size is small enough that you can fit $c + k$
rows in the cache, then you can keep the temporary buffer in the cache.  His reasoning is as follows:
\begin{itemize}
\item Phase 1: Only one row of the input is needed at a time, and the temporary buffer requires only $c+k-1$ rows.
\item Phase 2: $k$ rows of the temporary buffer are needed at a time to build up the output buffer of $c$ rows.
\end{itemize}
Assuming the cache uses an LRU replacement policy, and ignoring 
conflict misses, explain why this estimate may be too optimistic.

\begin{answer}
%% Write your answer to Problem 3A here



\end{answer}

\item
What do you propose as a safe rule for the chunk size?  (Some advice:
it's better to keep this simple and conservative.  If you have to run
a cache simulator to answer this question, you're overdoing it.)

\begin{answer}
%% Write your answer to Problem 3B here

\end{answer}

\item
How many rows can safely fit in the cache?  Remember that the
input rows must be padded with an extra $k-1$ pixels, and you want
to avoid false sharing.  Assume a cache block size of 64 bytes.  You can assume $k \leq 11$.

\begin{answer}
%% Write your answer to Problem 3C here

\end{answer}

\newpage

\item
When the caching works out, you find that the total amount of
work is a good predictor of the program execution time.  Give a formula,
in terms of $c$ and $k$, of how much work the chunked version must do,
relative to the baseline version.

\begin{answer}
%% Write your answers to Problem 3D here
\begin{description}
\item[Baseline:] 

\item[Chunked:]

\item[Ratio:]

\end{description} 
\end{answer}

\item
Consider the single-threaded case, and consider $k \in \{3,5,7\}$.
For each of these cases, what would be the maximum chunk size?
Give an estimate of the execution time, relative to
the baseline running on a machine with a very large
cache.

\begin{answer}
%% Write your answers to Problem 3E here
\begin{description}
\item[$k=3$:]

\item[$k=5$:]

\item[$k=7$:]

\end{description}
\end{answer}

\item
For values of $k \in \{3,5,7\}$, determine the number of threads that
would lead to optimum performance, considering that increasing the
number of threads requires decreasing the chunk size.  What would be
the chunk size?  Give an estimate of the execution time, relative to
the baseline running on a single-threaded machine with a very large
cache.

\begin{answer}
%% Write your answers to Problem 3F here
\begin{description}
\item[$k=3$:]

\item[$k=5$:]

\item[$k=7$:]

\end{description}
\end{answer}

\end{choice}


\end{document}
