%% You can use this file to create your answer for Exercise 1.  
%% Fill in the places labeled by comments.
%% Generate a PDF document by with the command `pdflatex ex1'.

\documentclass[11pt]{article}


\usepackage{alltt}
\usepackage{times}
\usepackage{listings}
\usepackage{enumerate}
\usepackage{courier}
\usepackage{hyperref}
\usepackage{xcolor}

\usepackage{booktabs}
\usepackage{graphicx}
\usepackage{color}

%% Values that are specific to a particular term
\newcommand{\thisterm}{Spring 2020}

\newcommand{\dateassigned}{Fri.,~Jan.~31}

%% Printed form of home page that students should use
\newcommand{\visiblecoursehome}{http://www.cs.cmu.edu/\textasciitilde{}418}

%% Link to home page that will stay valid
\newcommand{\actualcoursehome}{http://www.cs.cmu.edu/afs/cs.cmu.edu/academic/class/15418-s20/www}

\newcommand{\datedue}{Fri.,~Feb.~7}





%% Page layout
\oddsidemargin 0pt
\evensidemargin 0pt
\textheight 600pt
\textwidth 469pt
\setlength{\parindent}{0em}
\setlength{\parskip}{1ex}

%% Colored hyperlink 
\newcommand{\cref}[2]{\href{#1}{\color{blue}#2}}

%% Customization to listing
\lstset{basicstyle=\ttfamily\small,language=C,morekeywords={cilk_synch,cilk_spawn}}

%% Enumerate environment with alphabetic labels
\newenvironment{choice}{\begin{enumerate}[A.]}{\end{enumerate}}
%% Environment for supplying answers to problem
\newenvironment{answer}{\begin{minipage}[c][1.5in]{\textwidth}}{\end{minipage}}


\begin{document}
\begin{flushright}                         
{\large\bf Full Name: \makebox[2in][l]{    
%% Put your name on the next line          
                                           
                                           
}} \\[1ex]                                 
                                           
{\large\bf Andrew Id: \makebox[2in][l]{\tt 
%% Put your Andrew ID on the next line     
                                           
}} \\[1ex]                                 
\end{flushright}                           
\vspace*{0.3in}                            
\begin{center}
\LARGE
15-418/618 \thisterm{} \\
Exercise 2
\end{center}

\begin{center}
\Large        
\begin{tabular}{ll}
\hline             
Assigned: & \dateassigned{}  \\
Due: &  \datedue{}, 11:00~pm  \\
\hline       
\end{tabular}
\end{center} 

\section*{Overview}

This exercise is designed to help you better understand the lecture
material and be prepared for the style of questions you will get on
the exams.  The questions are designed to have simple answers.  Any
explanation you provide can be brief---at most 3 sentences.  You
should work on this on your own, since that's how things will be when
you take an exam.

You will submit an electronic version of this assignment to Gradescope 
as a PDF file.  For those of you familiar with the \LaTeX{} text 
formatter, you can download the template and configuation files at: 
\begin{center} 
  \cref{\actualcoursehome/exercises/config-ex2.tex}{\visiblecoursehome/exercises/config-ex2.tex}\\ 
  \cref{\actualcoursehome/exercises/ex2.tex}{\visiblecoursehome/exercises/ex2.tex} 
\end{center} 
Instructions for how to use this template are included as comments in 
the file.  Otherwise, you can use this PDF document as your starting 
point.  You can either: 1) electronically modify the PDF, or 2) print 
it out, write your answers by hand, and scan it.  In any case, we 
expect your solution to follow the formatting of this document. 

\section*{Problem 1: Instruction-Level Parallelism}

The following set of problems concern instruction-level parallelism
and the limitations to performance of loop code imposed by data
dependencies and resource limitations (as expressed by throughput bounds).

Calculating the distance between two vectors is an important operation for many
linear algebra packages and can be defined as:
\begin{displaymath}
\sqrt{\sum_{i=0}^{N-1}\left ( A\left [ i \right ]-B\left [ i \right ] \right )^{2}}.
\end{displaymath}

\newpage
Consider the following C code for calculating the squared distance 
between two vectors.

\begin{lstlisting}[language=C]
float distance(float A[], float B[], int N) {
	int i = 0;
	float total = 0;
	while (i < N) {                        // Performed by integer unit
		float a = A[i];                // Load
		float b = B[i];                // Load
		float diff = a - b;            // FP Add
		float squared = diff * diff;   // FP Multiply
		total = total + squared;       // FP Add
		++i;                           // Performed by integer unit
	}
	return total;
}
\end{lstlisting}


Suppose for the following questions you have a machine with multiple execution units of the following types:
\begin{description}
\item[Load/Store:] Performs load and store operations.  Can perform its own address arithmetic.
\item[Floating-point Adder:] Performs floating-point addition and subtraction
\item[Floating-point Multiplier:] Performs floating-point multiplication
\item[Integer:] Performs integer operations, including addition and comparison.
\end{description}

The processor has the following combination of execution units:
\begin{center}
\begin{tabular}{|l|c|c|}
\hline
\textbf{Unit}           & \textbf{Count} & \textbf{Latency} \\ \hline
Load/Store              & 1              & 4                \\ \hline
Floating-point Add      & 2              & 1                \\ \hline
Floating-point Multiply & 1              & 2                \\ \hline
Integer Add             & 2              & 1                \\ \hline
\end{tabular}
\end{center}
Each of the multi-cycle units is {\em fully pipelined}, able to begin a new operation on each clock cycle.

You should also assume the following:
\begin{itemize}
        \item The compiler produces optimized code.  All local variables are held in registers.
	\item The machine described has a sophisticated CPU with
	support for out-of-order execution, pipelining, branch prediction, and speculation.
	\item $N$ is very large ($> 10^{6}$).
	\item There are no cache misses.
        \item The only limits to average program performance are 1) the latencies due to data
          dependencies on loop-carried variables, and 2) the
          throughput limits of the functional units.
\end{itemize}

\begin{choice}
\item
Draw the dataflow diagram for the operations in the loop of \lstinline{distance}.  Use the local variable names as labels.
% Insert your diagram for 1A here.

\begin{figure}[h!]
  \centering
  \includegraphics*[width=0.6\linewidth]{1.A.png}
  \caption[]{The dataflow diagram}
\end{figure}

\newpage
\item
Which local variables create loop dependencies?

\begin{answer}
%% Put your answer to 1B here

Total and i create loop dependencies. 

\end{answer}

\item

What latency bound does the loop of \lstinline{distance} impose on the minimum average number of cycles for each element computed?
Justify your answer.

\begin{answer}
%% Put your answer to 1C here
\textcolor{blue}{A latency bound is encountered when a series of operations must be performed in strict sequence.}
\textcolor{red}{If there is some chain of data dependencies in a program where the sum of all of the latencies along that chain equals T, then the program will require at least T cycles to execute.}

Here, only the dependencies between iterations of the loop is considered. 
The addition of \lstinline{total} and \lstinline{squared}, and the increment of \lstinline{i} both create latency bounds of 1. These are the 
only two calculations that dependends on the previous iterations, and must be performed in sequence. 

\end{answer}

\item
What throughput bound does the set of available execution units impose on the loop of \lstinline{distance}?  Explain.

\begin{answer}
%% Put your answer to 1D here
\textcolor{blue}{A throughput bound characterizes the raw computing capacity of the processor's functional units.}
\textcolor{red}{Assume that a program requires a total of N computations of some operation, that the processor has C functional units capable of performing that operation, and that these units have an issue time of I. Then the program will require at least N · I/C cycles to execute.}
\textcolor{blue}{A program can achieve the throughput bound only when it can keep the pipelines filled for all of the functional units. 
}

The 2 Load instructions share a Load/Store unit, and thus create 2 cycles. All other functional units (FP Add, FP Mul, integer) create a bound of 1 cycle each. 

\end{answer}

\item
If you could add one additional execution unit to the mix above, what
would it be?  How would that change your program performance?

\begin{answer}
%% Put your answer to 1E here?
I would add another Load/Store Unit. This could reduce one cycle of the throughput bound. 
\textcolor{red}{And matches the latency bound. }

\end{answer}
\end{choice}


\section*{Barrier Synchronization}

The following set of problems are inspired by the code shown in Slides 48 and 49 in Lecture 05. 
In the following, we show only key parts of our version of the code. 
You can get the complete versions in the directory 
\begin{center} 
  \cref{\actualcoursehome/exercises/ex2-code}{\visiblecoursehome/exercises/ex2-code} 
\end{center} 
We suggest you download this code and study it.  You can also compile 
and run it.  (See the \texttt{README.txt} file.)  Insert print statements into the code to trace the 
actions of the different threads. 

The synchronization code in the slides show implement parts of a grid
solver.  To focus more directly on synchronization issues, we will
adapt the code for the following, highly contrived, application.

Let $B$ denote a {\em batch size} and $\theta$, with $0.0 < \theta <
1.0$ denote a {\em threshold.}  Suppose we perform a series of {\em
  phases}, where in each phase we compute $B$ random numbers, each ranging between 0.0 and 1.0, and take
their average $a$.  How many phases will it take to reach a case where
$a \leq \theta$, and what is the achieved value of $a$?  We assume the
``random'' numbers are actually generated by a pseudo-random
generator, and so, assuming we always start with the same seed, the
results will be deterministic.

Our implementation has $B$ threads running concurrently, using barrier
synchronization to keep them operating on the same phase.  The global variables are as follows:

\begin{lstlisting}
//// Global variables ////
// Read-only
float target_average; // Convergence goal
int batch_size;       // Number in batch
// Only written by single thread
long pcount = 0;      // Number of phases required
// Read-write  
float phase_sum;      // Sum of all values in current phase
\end{lstlisting}

\subsection*{A Three-Barrier Solution}


The following is the thread procedure for a version using three
barriers, similar to the code in Slide 48.  The full program is in the
file {\tt rconverge1.c}.  There are barriers between the three actions
performed on shared variable \texttt{phase\_sum} in each phase:
setting it to zero, incrementing by the random value, and testing for
convergence.  The call \texttt{uniform()} returns a (pseudo-)random
number between 0.0 and 1.0.

\newpage
\begin{lstlisting}
//// Thread procedure # 1 ////
void *thread_proc(void *ival) {
    long myid = (long) ival;
    bool myconverged = false;
    long count = 0;
    while (!myconverged) {
        count++;

        phase_sum = 0.0;                // Action 1: Set to zero

        barrier();   // Barrier #1

        float myval = uniform();
        atomic_add(&phase_sum,  myval); // Action 2: Increment

        barrier();   // Barrier #2

        myconverged =                   // Action 3: Test
            (phase_sum/batch_size) <= target_average;

        barrier();   // Barrier #3

    }
    if (myid == 0)
        pcount = count;
    return NULL;
}
\end{lstlisting}


\newpage
\subsection*{Problem 2: Understanding the Three-Barrier Solution}

For each of the three barriers, what can go wrong if you eliminate it?
You can try this with the actual code.  You may want to insert print
statements to track what happens.  Describe the behavior you observe,
and explain why it is happening.

\begin{choice}
\item Barrier \#1

\begin{answer}
%% Put your answer to 2A here
When barrier \#1 is eliminated, different threads will set \lstinline{phase_sum} to 0 at different time, which might cause the loss of random values that have already
added to \lstinline{phase_sum}. So the final \lstinline{phase_sum} is not the sum of all random values, and thus the program will converge easier. 
\end{answer}
\item Barrier \#2

\begin{answer}
%% Put your answer to 2B here

When barrier \#2 is eliminated, different threads judge \lstinline{myconverged} with different values of \lstinline{phase_sum} since they can reach the instruciton
at any time they want. So different thrads can have different values of \lstinline{myconverged}, and thus some threads might return before others. However, when some threads has not yet returned, they can never get off the barrier \#1
in the next phase, because there is not as much threads as required. And the whole program will keep waiting. 

\end{answer}

\item Barrier \#3

\begin{answer}
%% Put your answer to 2C here
When barrier \#2 is released, all threads have the same \lstinline{phase_sum}. If that phase have not yet converged, \lstinline{myconverged} should be \lstinline{false}. But without barrier \#3, some threads
might begin the next phase first, and reset \lstinline{phase_sum} to 0, and cause the threads that is still in the last phase get a wrong \lstinline{myconverged} due to a wrong \lstinline{phase_sum = 0}. And the rest is similar to the elimination of Barrier \#2. 
\end{answer}
\end{choice}

\subsection*{Single-Barrier Solutions}

As Slide 49 demonstrates, it is possible to transform the
three-barrier solution into one that uses only one barrier, using a
technique known as {\em software pipelining}.  The idea is to maintain
multiple versions of the shared, global variables, and make sure that
the different actions being performed within the loop operate on different versions.
The number of versions required is referred to as the {\em pipeline depth}.

For our case, we only need to have multiple copies of the variable \texttt{phase\_sum}:
\begin{lstlisting}
float phase_sum[PIPEDEPTH];  // Sum of all values in current phase
\end{lstlisting}
The quantity \texttt{PIPEDEPTH} is a compile-time constant.  In the
version provided to you, it is set to 128.

First, we will explore a version of the thread procedure that is
easier to understand.  The thread procedure is shown below.  The full
code is in the file \texttt{rconverge2.c}.  We see that it maintains
three indices into the \texttt{phase\_sum}, one for each of the basic
operations.  These indices are incremented by one for each phase,
modulo the pipeline depth.

\begin{lstlisting}
//// Thread procedure #2 ////
void *thread_proc(void *ival) {
    long myid = (long) ival;
    bool myconverged = false;
    int previndex = 0;
    int index = 1;
    long count = 0;
    phase_sum[1] = 0;
    barrier();       // Barrier #1

    while (!myconverged) {
	count++;

	int nextindex = (index+1) % PIPEDEPTH;
	phase_sum[nextindex] = 0.0;            // Action 1: Set to zero

	float myval = uniform();
	atomic_add(&phase_sum[index],  myval); // Action 2: Increment

        myconverged =                          // Action 3: Test
	    (phase_sum[previndex]/batch_size) <= target_average;

	barrier();   // Barrier #2

	previndex = index;
	index = nextindex;
    }
    if (myid == 0)
	pcount = count - 1;
    return NULL;
}
\end{lstlisting}

\newpage
\subsection*{Problem 3: Understanding the First Single-Barrier Solution}

\begin{choice}
\item Barrier \#1 is called only at the beginning of the procedure.
Describe its purpose.  What incorrect behavior can arise by eliminating it?

\begin{answer}
%% Put your answer to 3A here
The idea of this version of one barrier algorithm is that we check if the last phase is converged, calculate the sum of the current phase, and set the \lstinline{phase_sum} of the next phase to 0. 
Barrier \#2 is meant to make sure all threads go to the next phase together, since the current phase will overwrite the data of next phase. 

The purpose of barrier \#1 is to make sure all threads go to phase \#1 together. Without the barrier \#1, 
the slower threads will overwrite the addition of those faster threads when they executing \lstinline{phase_sum[1] = 0}. 

\end{answer}
\item Why does global variable \texttt{pcount} get set to \texttt{count - 1} in this code, but to \texttt{count} in the earlier code?

\begin{answer}
%% Put your answer to 3B here
Because the \lstinline{myconverged} test only test whether the last phase has converged, so if it passed, it means that the convergence is achieved in the last phase, instead of the current phase. 
\end{answer}
\item How small can constant \texttt{PIPEDEPTH} be set and still get
  correct behavior?  Explain why this is the case.

\begin{answer}
%% Put your answer to 3C here
The smallest \lstinline{PIPEDEPTH} could be 3. In each phase, we will concern about three phases, the previous one, the current one, and the next one. 
We only need to make sure that those three phases are in different position in \lstinline{phase_sum}. 
\end{answer}
\item Suppose you swap the order of Barrier \#2 with the code
  implementing Action \#3.
Explain why the program still works for the
  default version of \texttt{PIPEDEPTH}.

\begin{answer}
%% Put your answer to 3D here
When we do so, the threads might be different phase with each other when testing and setting to 0. 
Some threads are testing the previous phase of phase \lstinline{i}, while some are setting the next phase of phase \lstinline{i + 1} to zero. 
Namely, some threads are tesing phase \lstinline{i - 1}, while some are setting phase \lstinline{i + 2}. 

When the \lstinline{PIPEDEPTH} is 128, the subscripts of phase \lstinline{i - 1} and phase \lstinline{i + 2} are different, so there is no conflict. 
\end{answer}
\item
With this swapped version,
how small can you set \texttt{PIPEDEPTH} and still get the correct
answer?  Explain.

\begin{answer}
%% Put your answer to 3E here
We need to ensure that the subscripts of \lstinline{i - 1} and \lstinline{i + 2} is different. 

When the \lstinline{PIPEDEPTH} is 3, those two subscripts are the same, and thus conflict will be caused. 
If some threads testing after other threads set the \lstinline{phase_sum} to 0, they will return earlier, and leave other threads waiting in the barrier \#2 of next phase. 

The smallest number to do so is 4. 

\end{answer}
\end{choice}

\subsection*{Problem 4: Understanding the Second Single-Barrier Solution}
Our final version of the code resembles that seen in Slide 49:
\begin{lstlisting}
//// Thread procedure #3 ////
void *thread_proc(void *ival) {
    long myid = (long) ival;
    bool myconverged = false;
    int index = 1;
    long count = 0;
    phase_sum[1] = 0;
    barrier();       // Barrier #1

    while (!myconverged) {
	count++;

	int nextindex = (index+1) % PIPEDEPTH;
	phase_sum[nextindex] = 0.0;            // Action 1: Set to zero

	float myval = uniform();
	atomic_add(&phase_sum[index],  myval); // Action 2: Increment

	barrier();   // Barrier #2

        myconverged =                          // Action 3: Test
	    (phase_sum[index]/batch_size) <= target_average;

	index = nextindex;
    }
    if (myid == 0)
	pcount = count;
    return NULL;
}
\end{lstlisting}
\newpage
This version  maintains only two indices, and the
barrier synchronization has been put before Action \#3.  Let us explore this code:
\begin{choice}
\item Explain why the code does not require the third index \texttt{previndex}

\begin{answer}
%% Put your answer to 4A here
In the last version, we test on the previous phase because barrier \#2 is after the test, and we cannot ensure that all additions in this phase are finished when testing. 
However, in this version, all threads have passed barrier \#2 when testing, and thus we know that all additions have finished. 
So the \lstinline{phase_sum} of the current phase is the final sum wanted when testing. Thus, there is no need to check the current phase in the next phase. 
\end{answer}
\item In this version, the global variable \texttt{pcount} is set to
  the local value \texttt{count}, without decrementing it.  Explain
  why this is the correct result.

\begin{answer}
%% Put your answer to 4B here
As aforementioned, the testing is for the current phase, not the previous phase. So when \lstinline{myconverged = true}, that means the current phase is the answer. 
\end{answer}
\item How small can constant \texttt{PIPEDEPTH} be set and still get correct behavior?  Explain why this is the case.

\begin{answer}
%% Put your answer to 4C here
This is a little similar to 3D and 3E. 
There might exist a situation that the slower threads are testing the current phase of phase \lstinline{i} while others are setting the next phase of phase \lstinline{i + 1} to zero. 
Namely, some threads are testing the phase \lstinline{i} while others are setting phase \lstinline{i + 2} to 0. 

We need to ensure that the subscripts of phase \lstinline{i} and phase \lstinline{i + 2} have no conflict. So the smallest \lstinline{PIPEDEPTH} is 3. 
\end{answer}
\end{choice}

\newpage
\section*{Cilk Scheduling}

The following problems are based on the presentation of the Cilk
programming environment from Lecture 06.

Suppose we are running a program that uses the Cilk mechanisms for fork-join parallelism.  Assume the following:
\begin{itemize}
\item The system is running two threads
\item Every execution of {\tt cilk\_spawn} requires 1 millisecond.
  The executing thread will push its continuation at the top of its queue and
  begin executing (after 1ms) the spawned function (``child first''
  scheduling).
\item One thread can steal work from the queue of another.  It always
  steals from the bottom of the queue.  Stealing requires 2
  milliseconds, and then the thread can begin performing whatever operation is specified in the record.  
\item The procedure \texttt{foo} requires 3 milliseconds to complete.
\item All other operations require only a negligible amount of time.
\end{itemize}

As an example, consider the following code
\begin{lstlisting}
void simple() {
    cilk_spawn foo(1);
    foo(3);
    cilk_synch;
}
\end{lstlisting}

If Thread 1 starts executing \texttt{simple}, we can trace its execution as follows:
\begin{center}
\renewcommand{\arraystretch}{1.2}
\begin{tabular}{|c|l|l|}
\hline
Time & \makebox[2.75in]{Thread 1} & \makebox[2.75in]{Thread 2} \\
\hline
0
 &         %% T1
   Spawn \texttt{foo(1)}; Push \texttt{foo(3)} 
 &         %% T2 
   Idle
\\
\hline
1 
 &         %% T1
    Execute \texttt{foo(1)}
 &         %% T2 
    Steal \texttt{foo(3)}
\\
\hline
2 
 &         %% T1 
    Executing
 &         %% T2 
    Stealing
\\
\hline
3
 &         %% T1 
    Executing
 &         %% T2   
    Execute \texttt{foo(3)}
\\
\hline
4 
 &         %% T1 
    Idle
 &         %% T2  
    Executing
\\
\hline
5 
 & 
    Idle
 &
    Executing
\\
\hline
\end{tabular}
\end{center}

\newpage
\section*{Problem 5: Divide and Conquer Parallelism}

Consider the following function for executing multiple copies of function
\texttt{foo} via a series of forks that repeatedly split the problem
in half, analogous to the control structure of quicksort:
\begin{lstlisting}
void rfor(int start, int last) {
    if (start == last)
	foo(start);
    else {
	int middle = (start + last)/2;
	cilk_spawn rfor(start, middle);
	rfor(middle+1, last);
    }
    cilk_synch;
}
\end{lstlisting}

\begin{choice}
\item Fill in the following table, showing how two threads would
  handle the execution of \texttt{rfor(0,3)}.  (You may not require all of the rows in the table.)

%% Fill in the lines below for the table requested in Problem 5A.
%% See the formatting of the preceding table as an example
\begin{center}
\renewcommand{\arraystretch}{1.2}
\begin{tabular}{|c|l|l|}
\hline
Time & \makebox[2.75in]{Thread 1} & \makebox[2.75in]{Thread 2} \\
\hline
0 
 &                     %% T1
 Spawn rfor(0, 1), push rfor(2, 3)
 &                     %% T2
 Idle
\\
\hline
1
 &                     %% T1
 Spawn rfor(0, 0); push rfor(1, 1)
 &                     %% T2
 Steal rfor(2, 3)
\\
\hline
2
 &                      %% T1
 Executing foo(0)
 &                      %% T2
 Stealing
\\
\hline
3
 &                     %% T1
 Executing
 &                     %% T2
 Spawn rfor(2, 2); push rfor(3, 3)
\\
\hline
4
 &                     %% T1
 Executing
 &                     %% T2
 Executing foo(2)
\\
\hline
5
 &                     %% T1
 Executing foo(1)
 &                     %% T2
 Executing
\\
\hline
6
 &                     %% T1
 Executing
 &                     %% T2
 Executing
\\
\hline
7
 &                     %% T1
 Executing
 &                     %% T2
 Executing foo(3)
\\
\hline
8
 &                     %% T1
 Idle
 &                     %% T2
 Executing
\\
\hline
9
 &                     %% T1
 Idle
 &                     %% T2
 Executing
\\
\hline

\end{tabular}
\end{center}

\newpage
\item What do you get as the computed speedup for this execution?

\begin{answer}
%% Put your answer to 5B here
Serial execution takes $4\times3=12\ ms$, while parallel executiong take 10 ms as shown above. 

The speedup = $12 / 10 = 1.2$
\end{answer}
\end{choice}

\section*{Problem 6: Iterative Parallelism}

Consider the following function for executing multiple copies of
function \texttt{foo} by having the main thread spawning off series of
threads, each executing one instance of \texttt{foo}.  (Although the
function is written as a recursive procedure, you will see that the
sequence of spawns is identical what would occur if they were done as
a single loop, as in shown in Lecture 06, starting with Slides 39.)
\begin{lstlisting}
void ifor(int start, int last) {
    if (start == last)
	foo(start);
    else {
	cilk_spawn ifor(start, start);
	ifor(start+1, last);
    }
    cilk_synch;
}
\end{lstlisting}

\newpage
\begin{choice}
\item Fill in the following table, showing how two threads would
  handle the execution of \texttt{ifor(0,3)}:

%% Fill in the lines below for the table requested in Problem 6A.
%% See the formatting of the preceding table as an example
\begin{center}
\renewcommand{\arraystretch}{1.2}
\begin{tabular}{|c|l|l|}
\hline
Time & \makebox[2.75in]{Thread 1} & \makebox[2.75in]{Thread 2} \\
\hline
0 
 &                     %% T1
 Spawn ifor(0, 0); push ifor(1, 3)
 &                     %% T2
 Idle
\\
\hline
1
 &                     %% T1
 Execute foo(0)
 &                     %% T2
 Steal ifor(1, 3)
\\
\hline
2
 &                      %% T1
 Executing
 &                      %% T2
 Stealing
\\
\hline
3
 &                     %% T1
 Executing
 &                     %% T2
 Spawn ifor(1, 1); push ifor(2, 3)
\\
\hline
4
 &                     %% T1
 Steal ifor(2, 3)
 &                     %% T2
 Execute foo(1)
\\
\hline
5
 &                     %% T1
 Stealing
 &                     %% T2
 Executing
\\
\hline
6
 &                     %% T1
 Spawn ifor(2, 2); push ifor(3, 3)
 &                     %% T2
 Executing
\\
\hline
7
 &                     %% T1
 Execute foo(2)
 &                     %% T2
 Steal ifor(3, 3)
\\
\hline
8
 &                     %% T1
 Executing
 &                     %% T2
 Stealing
\\
\hline
9
 &                     %% T1
 Executing
 &                     %% T2
 Executing foo(3)
\\
\hline
10
 &                     %% T1
 Idle
 &                     %% T2
 Executing
\\
\hline
11
 &                     %% T1
 Idle
 &                     %% T2
 Executing
\\
\hline
\end{tabular}
\end{center}
\item What do you get as the computed speedup for this execution?

\begin{answer}
%% Put your answer to 6B here
The parallel execution takes 12 ms. So the speedup = $12 / 12 = 1.0$
\end{answer}

\item What insights do these to example give you regarding the best way to use Cilk in exploiting fork-join parallelism?

\begin{answer}
%% Put your answer to 6C here
Stealing is costly, so we want to reduce stealing as much as possible. 
Compare to decrease-and-conquer in problem 6, divide-and-conquer in problem 5 is better. 
\end{answer}
\end{choice}



\end{document}
