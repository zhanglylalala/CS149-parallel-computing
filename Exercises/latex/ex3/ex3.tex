%% You can use this file to create your answer for Exercise 1.  
%% Fill in the places labeled by comments.
%% Generate a PDF document by with the command `pdflatex ex1'.

\documentclass[11pt]{article}


\usepackage{times}
\usepackage{listings}
\usepackage{enumerate}
\usepackage{courier}
\usepackage{hyperref}
\usepackage{xcolor}
\usepackage{graphicx}


%% Values that are specific to a particular term
\newcommand{\thisterm}{Spring 2020}

\newcommand{\dateassigned}{Mon.,~Feb.~17}

%% Printed form of home page that students should use
\newcommand{\visiblecoursehome}{http://www.cs.cmu.edu/\textasciitilde{}418}

%% Link to home page that will stay valid
\newcommand{\actualcoursehome}{http://www.cs.cmu.edu/afs/cs.cmu.edu/academic/class/15418-s20/www}

\newcommand{\datedue}{Fri.,~Feb.~21}





%% Page layout
\oddsidemargin 0pt
\evensidemargin 0pt
\textheight 600pt
\textwidth 469pt
\setlength{\parindent}{0em}
\setlength{\parskip}{1ex}

%% Colored hyperlink 
\newcommand{\cref}[2]{\href{#1}{\color{blue}#2}}

%% Customization to listing
\lstset{basicstyle=\ttfamily\small,language=C,morekeywords={cilk_synch,cilk_spawn}}

%% Enumerate environment with alphabetic labels
\newenvironment{choice}{\begin{enumerate}[A.]}{\end{enumerate}}
%% Environment for supplying answers to problem
\newenvironment{answer}{\begin{minipage}[c][1.5in]{\textwidth}}{\end{minipage}}


\begin{document}
\begin{flushright}                         
{\large\bf Full Name: \makebox[2in][l]{    
%% Put your name on the next line          
                                           
                                           
}} \\[1ex]                                 
                                           
{\large\bf Andrew Id: \makebox[2in][l]{\tt 
%% Put your Andrew ID on the next line     
                                           
}} \\[1ex]                                 
\end{flushright}                           
\vspace*{0.3in}                            
\begin{center}
\LARGE
15-418/618 \thisterm{} \\
Exercise 3
\end{center}

\begin{center}
\Large        
\begin{tabular}{ll}
\hline             
Assigned: & \dateassigned{}  \\
Due: &  \datedue{}, 11:00~pm  \\
\hline       
\end{tabular}
\end{center} 

\section*{Overview}

This exercise is designed to help you better understand the lecture
material and be prepared for the style of questions you will get on
the exams.  The questions are designed to have simple answers.  Any
explanation you provide can be brief---at most 3 sentences.  You
should work on this on your own, since that's how things will be when
you take an exam.

You will submit an electronic version of this assignment to Gradescope 
as a PDF file.  For those of you familiar with the \LaTeX{} text 
formatter, you can download the template and configuation files at: 
\begin{center} 
  \cref{\actualcoursehome/exercises/config-ex3.tex}{\visiblecoursehome/exercises/config-ex3.tex}\\ 
  \cref{\actualcoursehome/exercises/ex3.tex}{\visiblecoursehome/exercises/ex3.tex} 
\end{center} 
Instructions for how to use this template are included as comments in 
the file.  Otherwise, you can use this PDF document as your starting 
point.  You can either: 1) electronically modify the PDF, or 2) print 
it out, write your answers by hand, and scan it.  In any case, we 
expect your solution to follow the formatting of this document. 


\section*{Problem 1: GPU Programming}

Your friends have been hearing that you now know how to program a GPU
and take advantage of all its capabilities. Eager to bask in your
knowledge, they come to you with different problems that they think
might benefit from GPU parallelism.

One such friend heard that GPUs can launch thousands of parallel
``threads.'' They want to identify the indices of a large array of
$n$ elements in an even larger array (of size $m$) using binary search. They suggest to
you that every thread on the GPU will be searching for one element in
the array, so that with say, 1024 threads, they'll be able to search
get near $1024\times$ performance.

\begin{choice}
\item
First off, how do you explain that
threads on a CPU are not the same as threads on a GPU? (What's the
difference between the implementation of p\_threads and CUDA threads?)
Give at least 3 important distinctions.


\begin{answer}
1. Different p\_threads can execute different instructions while CUDA threads are SPMD which
means that they are executing the same instructions on different data. 

2. The synchronization and scheduling of CUDA threads are done by hardware, while those of 
p\_threadsare done by software. 

3. CUDA threads are mapped restrictly into certain blocks and store data in execution context
in their block. Before every execution, 32 data are chosen to fill all warps and execute the 
same instructions. 


\end{answer}

\item
Still convinced of their binary search's potential, they write this kernel function.

\begin{lstlisting}[language=C,basicstyle=\ttfamily]
// Array S and R are of size n
// Array A is of size m, with its elements ordered.
__global__ void cudaBinarySearch(int n, int m, int *S, int *A, int *R) {
    int i = blockDim.x * blockIdx.x + threadIdx.x;
    if (i >= n) return;
    
    int search = S[i];
    int left = 0, right = m - 1, middle = 0;
    R[i] = -1;  // In case not found
    // Binary Search
    while (left <= right) {
      middle = left + (right - left) / 2;      
      if (search < A[middle])
        right = middle - 1;
      else if (search > A[middle])
        left = middle + 1;
      else
        R[i] = middle;
        break;
    }
}
\end{lstlisting}

After testing it out, they find that they get very little performance
gain relative to the number of threads that they are running. Sullenly,
they come back to you and ask where they went wrong. Give at least
two reasons for this code's poor performance.

\begin{answer}
1. The kernel function has too many control flow, like while, if-else. All CUDA threads in
the same block are executing the same instructions, which causes they have to execute some extra
instructions and mask some unwanted results to ensure all threads have their results. 
2. It read frequently from global memory, caused by A[middle]. 
\end{answer}
\end{choice}

\section*{Problem 2: Problem Scaling}

This problem is a three-dimensional extension of the grid problem
described in the lecture on workload-driven performance evaluation:
\begin{center}
\cref{\actualcoursehome/lectures/10_perfeval.pdf}{\visiblecoursehome/lectures/10\_perfeval.pdf}.
\end{center}


Consider an iterative solver that operates on a three-dimensional grid
of size $N \times N \times N$.  It requires
$N$ iterations to reach convergence.  We refer to
the parameter $N$ as the {\em problem size}.  Increasing $N$ by a
factor of 2 increases the number of grid elements 8$\times$ and the number
of iterations 2$\times$.

The state of the system is represented as a three-dimensional array of
values {\tt state}.  The function {\tt update} computes a new value of
the state based on the current state.  For each element {\tt
  state[x][y][z]}, {\tt update} computes the new value as a function of its current value and that
of its six neighbors:
\begin{description}
\item[Self:]  {\tt state[x][y][z]}
\item[West:] {\tt state[x-1][y][z]}
\item[East:] {\tt state[x+1][y][z]}
\item[North:] {\tt state[x][y-1][z]}
\item[South:] {\tt state[x][y+1][z]}
\item[Down:] {\tt state[x][y][z-1]}
\item[Up:] {\tt state[x][y][z+1]}
\end{description}
(You don't need to know the exact function.)

The overall operation can then be described in a pseudo-C notation as:
\begin{lstlisting}[language=C,basicstyle=\ttfamily]
// For each iteration
for (iter = 0; iter < N; iter++) {
   // Main computation
   for all x, y, z in 0..N
       nstate[x][y][z] = update(state, x, y, z);
   for all x, y, z in 0..N
       state[x][y][z] = nstate[x][y][z]
}
\end{lstlisting}

In mapping the computation onto $P$ processors, the state array is
split into $P$ cubic blocks, each containing $N^3/P$ array elements.
Each iteration (labeled ``Main computation'') involves
having each processor 1) communicate the boundary values with its (up
to six) neighbors, and 2) computing the update function for all
$N^3/P$ of its assigned elements.  The program uses $M$ gigabytes of
memory per processor and runs in total time $T$.

\begin{figure}
\begin{center}
% \includegraphics[width=6in]{figs/cubes.pdf} %% QUES
Figure omitted
\end{center}
\caption{Scaling a problem by doubling the grid size $N$ (a), or by increasing the number of processors  8$\times$ (b).}
\label{fig:scale}
\end{figure}

\begin{enumerate}
\item Consider the following two scaling possibilities, yielding a problem of size $N'$
  running on $P'$ processors, as illustrated in Figure \ref{fig:scale}.
\begin{description}
\item[(a)] Double the value of $N$, while holding $P$ constant:  $N' = 2N$, $P' = P$
\item[(b)] Hold $N$ constant, while increasing $P$ by 8$\times$: $N' = N$, $P' = 8P$
\end{description}

Fill in the table below showing how the amounts of computation and
communication for a single processor, performing a single iteration
would scale, relative to the original problem.  (For example, $2\times$ would indicate growth by a
factor of two.)
\begin{center}
\renewcommand{\arraystretch}{2}
\begin{tabular}{|l|c|c|}
\hline
& \makebox[1.5in][c]{Computation} & \makebox[1.5in][c]{Communication} \\
\hline
(a): $N' = 2N$, $P' = P$ & $8\times$
& $4\times$
\\
\hline
(b): $N' = N$, $P' = 8P$ & $\frac{1}{8}\times$
& $\frac{1}{4}\times$
\\
\hline
\end{tabular}
\end{center}
\item 
Now assume both $N$ and $P$ can vary.  Based on these specific cases,
give formulas for how the computation, communication, and arithmetic
intensity would scale as functions of $N$ and $P$.  Again, quantify
these values with respect to the computation and communication each
processor would perform during a single iteration.  (By way of
reference, the formulas for the two-dimensional version were given in
slide 7 of the lecture.)
\begin{center}
\renewcommand{\arraystretch}{2}
\begin{tabular}{|c|c|c|}
\hline
 \makebox[1.5in]{Computation} & \makebox[1.5in]{Communication} & \makebox[1.5in]{Arithmetic Intensity}\\
\hline
$\frac{N^3}{P}$ & $\frac{N^2}{P^{2/3}}$ & $\frac{N}{P^{1/3}}$
 \\
\hline
\end{tabular}
\end{center}
\item  Suppose we have a machine with $8P$ processors, each
  identical to the original $P$ processors.  We consider three scaling
  possibilities, yielding a new problem of size $N'$ requiring
  total time $T'$ and $M'$ gigabytes per processor:
\begin{description}
\item[Problem scaling:] $N' = N$. Solve the same problem faster.  Goal is to minimize $T'$
\item[Memory scaling:] $M' = M$.  Make full use of the increased total memory.  Goal is to maximize $N'$.
\item[Time scaling:] $T' = T$.  Solve a bigger problem in the same amount of time.  Goal is to maximize $N'$.
\end{description}
Fill in the following table with formulas indicating the problem size
$N'$, the per-processor memory requirement $M'$, the ideal total time
$T'$, and the change in arithmetic intensity.
(By way of reference, this information for the two-dimensional version was given in slides 17, 22, and 25 of the lecture.)
\begin{center}
\renewcommand{\arraystretch}{2}
\begin{tabular}{|c|c|c|c|c|}
\hline
Scaling Type & \makebox[.6in]{$N'$} & \makebox[.6in]{$M'$} & \makebox[.6in]{$T'$} & \makebox[1in]{Arith.~Intensity} \\
\hline
Problem & $N$ & $\frac{1}{8}M$ & $\frac{1}{8}T$ & $\frac{1}{2}\times$  \\
\hline
Memory & $2N$ & $M$ & $2T$ & $1\times$  \\
\hline
Time & $2^{3/4}N$ & $2^{-3/4}M$ & $T$ & $2^{-1/4}\times$ \\
\hline
\end{tabular}
\end{center}
\end{enumerate}  

\section*{Problem 3: Cache coherency}

  Your friend suggests modifying the MSI coherence protocol so that PrRd / BusRd behavior on the
  I-to-S transition is changed to PrRd / BusRdX, as is shown below:

\begin{center}
% \includegraphics[width=3in]{figs/msi.pdf} %% QUES
Figure omitted
\end{center}


\begin{choice}

\item
  Is the memory system still coherent? What impact does this change have on the system?

\begin{answer}
The memory system is still conherent. 

Even if a processor is only going to read the line, instead of writing it, other processors with the 
line will turn to invalid. Even worse, it will flush those in modified state. 
So this change will lower the efficiency of the overall system. 

\end{answer}

\newpage
\item 
  You are hired by Intel to design an invalidation based cache-coherent processor with a large number of cores.
The 
  main application for this processor will be to train a chess AI by playing games against itself for experiments. It is critical that it
  doesn't play too many games during training to maintain the validity of the experiment, so we store how many
  games are played in a shared counter:
  \begin{lstlisting}
  int num_games = 100000;
  int games_played = 0;

  // This code is run on each core
  while (true) {
    // Atomically get value of count prior to increment, and
    // write incremented value to games_played
    int val = atomic_add(&games_played, 1);   

    if (val > num_games) break;
    play_game()
  }
  \end{lstlisting}

  Unsure on how to design the cache coherence of this processor, you check with one of the senior designers. She suggests that you 
  should use a bus-based, snooping coherence implementation, rather than a directory-based protocol, because 
 the broadcasting it performs would be more
  efficient in this case. Do you agree or disagree? Why or why not?

\begin{answer}
  I agree with her. In this program, \lstinline{num_games} won't change and always stays in cache. 
  But \lstinline{game_played} will be modified frequently. If the system is directory-based, 
  most of the communication will involve the home node, which makes the workload imbalanced. 

  \textcolor{red}{
    Disagree, this program has frequent writes so only one core will have the shared counter 
    in cache at a time. Thus a directory would be better as it would only have to invalidate 
    one other cache line, rather than broadcast to all.
  }

  \textcolor{cyan}{
    The cache coherence is controlled by cache controller, not processor. So the heavy communication
    on a singal node won't affect program speed significantly. 
  }
\end{answer}

\newpage
\item 
  You are now tasked with finding the best performing chess AI. You plan to run the following code on the
  processor:
  \begin{lstlisting}
  int num_experiments = 10000;
  int best_score = 0;

  #pragma omp parallel for 
  for (int i = 0; i < num_experiments; i++) {
    int score = perform_experiment(i);
    if (score > best_score) {
      // Atomically retrieve best score, compute max,
      // and write result to best_score
      atomic_max(&best_score, score);
    }
  }
  \end{lstlisting}
  Trying to improve the performance of this code, you remove the \texttt{if} statement prior to the \verb|atomic_max|,
  noticing that it is redundant. Suprisingly, the performance of the program is drastically reduced. Why did
  this happen?

\begin{answer}
  Without the \lstinline{if} statement, all processors will execute the \lstinline{atomic_max} in
  sequence. But with the \lstinline{if} statement, only those procrssors with \lstinline{score > best_score} 
  will execute the \lstinline{atomic_max} sequentially. 
  \textcolor{red}{
    Remember that fetch and max always causes a write. By removing the if statement, every 
    iteration per- forms a write even if no write was needed. This causes each iteration to take 
    a cache miss on best score.
  }
\end{answer}

\newpage
\item 
  You are hired as a TA for 15-418/618.  In an attempt to speed up
  exam grading, you decide to multithread the grading
  program on a cache coherent system as shown below
  \begin{lstlisting}
  int scores[NUM_STUDENTS];

  #pragma omp parallel for schedule(dynamic)
  for (int student = 0; student < NUM_STUDENTS; student++) {
      for (int problem = 0; problem < NUM_EXAM_PROBLEMS; problem++) {
          // performs a write to scores[student] 
          grade_and_update(student, problem, &scores[student]); 
      }
  }
  \end{lstlisting}
  Dismayed by the speed of this program, you analyze the runtime of this program and find that there
  is a large amount of bus communication during the execution.

  What kind of communication is happening in this program? What performance problem is this a result of?

\begin{answer}
  \textcolor{red}{
    There is a large amount of artifactual communication happening in this program. As multiple 
    student’s scores occupy the same cache line, there is a lot of false sharing requiring large 
    amounts of communication due to the coherence protocol.
  }
\end{answer}

\end{choice}  


\end{document}
