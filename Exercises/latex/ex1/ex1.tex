%% You can use this file to create your answer for Exercise 1.  
%% Fill in the places labeled by comments.
%% Generate a PDF document by with the command `pdflatex ex1'.

\documentclass[11pt]{article}


\usepackage{times}
\usepackage{listings}
\usepackage{enumerate}
\usepackage{courier}
\usepackage{hyperref}
\usepackage{xcolor}

\usepackage{booktabs}
\usepackage{color}
\usepackage{multicol}

%% Values that are specific to a particular term
\newcommand{\thisterm}{Spring 2020}

\newcommand{\dateassigned}{Fri.,~Jan.~17}

%% Printed form of home page that students should use
\newcommand{\visiblecoursehome}{http://www.cs.cmu.edu/\textasciitilde{}418}

%% Link to home page that will stay valid
\newcommand{\actualcoursehome}{http://www.cs.cmu.edu/afs/cs.cmu.edu/academic/class/15418-s20/www}

\newcommand{\datedueregistered}{Fri.,~Jan.~24}
\newcommand{\dateduewaitlist}{Fri.,~Jan..~31}




%% Page layout
\oddsidemargin 0pt
\evensidemargin 0pt
\textheight 600pt
\textwidth 469pt
\setlength{\parindent}{0em}
\setlength{\parskip}{1ex}

%% Colored hyperlink 
\newcommand{\cref}[2]{\href{#1}{\color{blue}#2}}

%% Customization to listing
\lstset{basicstyle=\ttfamily,language=C++,morekeywords={uniform,foreach}}

%% Enumerate environment with alphabetic labels
\newenvironment{choice}{\begin{enumerate}[A.]}{\end{enumerate}}
%% Environment for supplying answers to problem
\newenvironment{answer}{\begin{minipage}[c][1.5in]{\textwidth}}{\end{minipage}}

\begin{document}
\begin{flushright}
{\large\bf Full Name: \makebox[2in][l]{
%% Put your name on the next line

}} \\[1ex]

{\large\bf Andrew Id: \makebox[2in][l]{\tt
%% Put your Andrew ID on the next line

}} \\[1ex]
\end{flushright}
\vspace*{0.3in}
\begin{center}
\LARGE
15-418/618 \thisterm{} \\
Exercise 1
\\ 
\end{center}

\begin{center}
\Large
\begin{tabular}{lll}
\hline
 & Registered students & Waitlist students \\
\hline
Assigned: & \dateassigned{} & \dateassigned{} \\
Due: &  \datedueregistered{}, 11:00~pm & \dateduewaitlist{}, 11:00~pm \\
\hline
\end{tabular}
\end{center}

\section*{Overview}

This exercise is designed to help you better understand the lecture
material and be prepared for the style of questions you will get on
the exams.  The questions are designed to have simple answers.  Any
explanation you provide can be brief---at most 3 sentences.
You should work on this on your own, since that's how things will be when
you take an exam.

You will submit an electronic version of this assignment to Gradescope as
a PDF file.  For those of you familiar with the \LaTeX{} text formatter, you can download the template and configuration files at:
\begin{center}
  \cref{\actualcoursehome/exercises/config-ex1.tex}{\visiblecoursehome/exercises/config-ex1.tex}\\
  \cref{\actualcoursehome/exercises/ex1.tex}{\visiblecoursehome/exercises/ex1.tex}
\end{center}
Instructions for how to use this template are included as comments in the file.  Otherwise,
you can use this PDF document as your starting point.
You can either: 1) electronically modify the PDF, or 2) print it
out, write your answers by hand, and scan it.  In any case, we expect
your solution to follow the formatting of this document.

\newpage

\section*{Problems}

Consider the following code where each line within the function
represents a single instruction.
\begin{lstlisting}
typedef struct {
    float x;
    float y;
} point;

inline void innerProduct(point *a, point *b, float *result)
{
    float x1 = a->x; // Uses a load instruction
    float x2 = b->x;
    float product1 = x1*x2;
    float y1 = a->y;
    float y2 = b->y;
    float product2 = y1*y2;
    float inner = product1 + product2;
    *result = inner; // Uses a store instruction
}

void computeInnerProduct(point A[], point B[], float result[], int N)
{
    for (int i = 0; i < N; i++)
        innerProduct(&A[i], &B[i], &result[i]);
}
\end{lstlisting}

In the following questions,
you can assume the following:
\begin{itemize}
\item $N$ is very large ($> 10^6$).
\item The machines described have modern CPUs, providing
out-of-order execution, speculative execution, branch prediction, etc.
\begin{itemize}
\item There are ample resources for fetching, decoding, and committing instructions.  The only performance limitations are due to
the number, capabilities, and latencies of the execution units.
\item The branch prediction is perfect.
\end{itemize}
\item There are no cache misses.
\item The overhead of updating
the loop index {\tt i} is negligible.
\item The load/store units perform any necessary address arithmetic.
\item The overhead due to procedure calls, as well as starting and ending loops, is negligible.
\end{itemize}

\newpage

\section*{Problem 1: Instruction-Level Parallelism}

Suppose you have a machine $M_1$ with two load/store units
  that can each load or store a single value on each clock cycle, and
  one arithmetic unit that can perform one arithmetic operation
  (e.g., multiplication or addition) on each clock cycle.


\begin{choice}
\item  
Assume that the load/store and arithmetic units have latencies of one cycle.
How many clock cycles would be required to
execute \texttt{computeInnerProduct} as a function of $N$?    Explain what limits the performance.

\begin{answer}
%% Enter your answer to Problem 1A here
%% Write formulas in math mode.  E.g., $N/17.4$ or $N/\sqrt{14}$.

3N. The flow of instructions executed by each unit is shown in the table below. \\
The limitation is the number of arithmetic units. 

\end{answer}
\item 
Now assume that the load/store and arithmetic unit have
latencies of 10 clock cycles, but they are fully pipelined, able to
initiate new operations every clock cycle.  
How many clock cycles would be required to
execute \texttt{computeInnerProduct} as a function of $N$?    Explain how this relates to your answer to part A.

\begin{answer}
%% Enter your answer to Problem 1B here
3N. No loop dependencies here, so we can still launches all instructions in order, and use 
multiple registers to store the temprary results. Thus the times isn't changed. \\
In part A, due to the short latencies, it only needs one registers each for x1, x2, y1, y2, product1, product2, and inner. But part B needs more. 

\end{answer}
\end{choice}

\begin{tabular} {ccc}
  \toprule
  Arith Unit & L/S Unit1 & L/S Unit2 \\
  \midrule
  & \textcolor{blue}{x1 = a[1].x} & \textcolor{blue}{x2 = b[1].x} \\
  \textcolor{blue}{MUL x1 x2} & \textcolor{blue}{y1 = a[1].y} & \textcolor{blue}{y2 = b[1].y} \\
  \textcolor{blue}{MUL y1 y2} & \textcolor{magenta}{x1 = a[2].x} & \textcolor{magenta}{x2 = b[2].y} \\
  \textcolor{blue}{ADD} & \textcolor{magenta}{y1 = a[2].y} & \textcolor{magenta}{y2 = b[2].y}\\
  \textcolor{magenta}{MUL x1 x2} & \textcolor{blue}{Store} & \\
  \textcolor{magenta}{MUL y1 y2} & x1 = a[3].x & x2 = b[3].x\\
  \textcolor{magenta}{ADD} & y1 = a[3].y & y2 = b[3].y\\
  MUL x1 x2 & \textcolor{magenta}{Store} & \\
  MUL y1 y2\\
  ADD\\
  &Store\\
  \bottomrule

\end{tabular}

\newpage
\section*{Problem 2: SIMD with ISPC}


Consider running the following ISPC code.
\begin{lstlisting}
export void computeInnerProductISPC(uniform point[] A, 
                                    uniform point[] B,
                                    uniform float[] result,
                                    uniform int N)
{
    foreach(i = 0 ... N)
    {
        result[i] = A[i].x * B[i].x + A[i].y * B[i].y;
    }
}
\end{lstlisting}

Suppose machine $M_2$ has one 8-wide SIMD load/store unit,
  and one 8-wide SIMD arithmetic unit.  Both have latencies of one clock cycle.

\begin{choice}
\item
  How
  many clock cycles would be required to execute
  \texttt{computeInnerProductISPC} as a function of $N$?
  Explain what limits the performance.

\begin{answer}
%% Enter your answer to Problem 2A here

$\frac{5}{8}N$. Just like Problem 1A, the limitation is the number of arithmetic units.  \\
\begin{tabular}{cc}
  \toprule
  SIMD load/store & SIMD Arith\\
  \midrule
  Load .x\\
  Load .y & MUL\\
  &MUL\\
  &ADD\\
  Store\\
  \bottomrule
\end{tabular}

\end{answer}

\item  If we were to run the code shown in \texttt{computeInnerProductISPC} on a five-core
  machine $M_3$, where each core has the same SIMD capabilities as $M_2$,
  what would be the best speedup it could achieve over the single-core performance of part A?  Explain.

\begin{answer}
%% Enter your answer to Problem 2B here
%% When writing speedups, use the math symbol \times, rather than simply X.
%% e.g., $3.14 \times$.
Only $1\times$. \textcolor{blue}{It would only have one thread, thus only can use a signle core. }

\end{answer}
\end{choice}

\newpage
\section*{Problem 3: SIMD, Multicore, and Multi-Threaded Parallelism with ISPC}

\begin{choice}
\item
Consider the five-core machine $M_3$ described in Problem 2B\@.
Suppose you could write multi-threaded code where there is no overhead associated with the threads.  Each thread would run the function \texttt{computeInnerProductISPC} to compute some subset of the $N$ elements.
What is the maximum speedup you could achieve relative to the single-threaded code running on machine $M_2$?

\begin{answer}
%% Enter your answer to Problem 3A here
$5\times$. To achieve the maximum speedup, it could create 5 threads to take advantage of all 5 cores. 

\end{answer}


\item
Now suppose we have a machine $M_4$, identical to $M_3$, except that
each core supports three-way simultaneous multithreading.  What is the
maximum speedup your multithreaded code could achieve relative to what it achieved running
on machine $M_3$.

\begin{answer}
%% Enter your answer to Problem 3B here
$1\times$. Multithreading can be used to hide the latencies, but there is no latency here. 

\end{answer}

\item
Let $N = 10^6$.
Running on machine $M_3$, if we were to write a Pthreads program that spawns
  250 threads, each computing 4000 inner products using
  \texttt{computeInnerProductISPC}, would this program get an overall
  performance improvement over one that uses a single thread to compute all $N$ inner products?  
(Use your own intution about the cost of spawning new threads in Pthreads when answering this question.)
Explain.

\begin{answer}
%% Enter your answer to Problem 3C here
Yes. \\
If there is only one single thread, only one core is used. \\
However, with 250 threads, all 5 cores are used. So the time spent on executing instructions is 5 times less. 
Even consider the cost of switching threads, the gap is still huge. \\ \\
\textcolor{red}{
  No. \\
  Each thread would use only around $4000\times \frac{5}{8}=2500$ cycles for the computation. 
  Yet the thread overhead is way more than 7500 cycles (usually takes several $\mu s$)
}

\end{answer}

\newpage
\item
Let $N = 10^6$.
Running on machine $M_3$, if we were to write an ISPC program that launches
  250 tasks, each computing 4000 inner products using
  \texttt{computeInnerProductISPC}, would this program get an overall
  performance improvement over one that uses a single task to compute all $N$ inner products?
(Consider what you know about the performance characteristics of ISPC tasks.)
Explain.

\begin{answer}
%% Enter your answer to Problem 3D here
Yes. \\ \\
\textcolor{red}{
  ISPC task is more efficient than threads. That each task uses 2500 cycles
  is still worthwhile for tasking. However, the overhead still exists. 
}

\end{answer}
\end{choice}

\end{document}
